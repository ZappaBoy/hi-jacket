\documentclass[
11pt, % The default document font size, options: 10pt, 11pt, 12pt
oneside, % Two side (alternating margins) for binding by default, uncomment to switch to one side
english, % ngerman for German
onehalfspacing,%singlespacing, % Single line spacing, alternatives: onehalfspacing or doublespacing
%draft, % Uncomment to enable draft mode (no pictures, no links, overfull hboxes indicated)
%nolistspacing, % If the document is onehalfspacing or doublespacing, uncomment this to set spacing in lists to single
%liststotoc, % Uncomment to add the list of figures/tables/etc to the table of contents
%toctotoc, % Uncomment to add the main table of contents to the table of contents
%parskip, % Uncomment to add space between paragraphs
%nohyperref, % Uncomment to not load the hyperref package
headsepline, % Uncomment to get a line under the header
%chapterinoneline, % Uncomment to place the chapter title next to the number on one line
%consistentlayout, % Uncomment to change the layout of the declaration, abstract and acknowledgements pages to match the default layout
]{MastersDoctoralThesis} % The class file specifying the document structure

\usepackage[utf8]{inputenc} % Required for inputting international characters
\usepackage[T1]{fontenc} % Output font encoding for international characters
\usepackage{mathpazo} % Use the Palatino font by default
\usepackage[backend=bibtex,style=authoryear,natbib=true]{biblatex} % Use the bibtex backend with the authoryear citation style (which resembles APA)
\usepackage[autostyle=true]{csquotes} % Required to generate language-dependent quotes in the bibliography
\usepackage{enumitem}
\usepackage{listings}
\usepackage{graphicx}
\usepackage{xcolor}
\usepackage{amsmath}
\usepackage[font={small}]{caption}
\usepackage{varwidth}
\usepackage{supertabular}
\usepackage[caption=false]{subfig}
\usepackage{wrapfig}
\usepackage{mathtools}

\addbibresource{related_work.bib} % The filename of the bibliography

\graphicspath{
    {Figures/}
}

\newenvironment{blueparagraph}{\par\color{blue}}{\par}
\newenvironment{nscenter}
 {\parskip=0pt\par\nopagebreak\centering}
 {\par\noindent\ignorespacesafterend}

%----------------------------------------------------------------------------------------
%	MARGIN SETTINGS
%----------------------------------------------------------------------------------------

\geometry{
	paper=a4paper, % Change to letterpaper for US letter
	inner=2.5cm, % Inner margin
	outer=3.8cm, % Outer margin
	bindingoffset=.5cm, % Binding offset
	top=1.5cm, % Top margin
	bottom=1.5cm, % Bottom margin
	%showframe, % Uncomment to show how the type block is set on the page
}

%----------------------------------------------------------------------------------------
%	THESIS INFORMATION
%----------------------------------------------------------------------------------------

\thesistitle{TODO GPU cryptojacking} % Your thesis title, this is used in the title and abstract, print it elsewhere with \ttitle
%\supervisor{Dr.\ Name \textsc{Surname}} % Your supervisor's name, this is used in the title page, print it elsewhere with \supname
%\examiner{} % Your examiner's name, this is not currently used anywhere in the template, print it elsewhere with \examname
%\degree{Computer Science} % Your degree name, this is used in the title page and abstract, print it elsewhere with \degreename
\author{Federico \textsc{Zappone}} % Your name, this is used in the title page and abstract, print it elsewhere with \authorname
%\addresses{} % Your address, this is not currently used anywhere in the template, print it elsewhere with \addressname

\subject{Networking security and software security} % Your subject area, this is not currently used anywhere in the template, print it elsewhere with \subjectname
\keywords{GPU, cryptojacking, xss, defense} % Keywords for your thesis, this is not currently used anywhere in the template, print it elsewhere with \keywordnames
\university{\href{https://www.unimol.it/}{University of Molise}} % Your university's name and URL, this is used in the title page and abstract, print it elsewhere with \univname
\department{\href{http://dipbioter.unimol.it/}{Departement of Bioscience and Territory}} % Your department's name and URL, this is used in the title page and abstract, print it elsewhere with \deptname
%\faculty{\href{http://faculty.university.com}{Faculty Name}} % Your faculty's name and URL, this is used in the title page and abstract, print it elsewhere with \facname

\AtBeginDocument{
\hypersetup{pdftitle=Title\ttitle} % Set the PDF's title to your title
\hypersetup{pdfauthor=Federico Zappone\authorname} % Set the PDF's author to your name
\hypersetup{pdfkeywords=keyword1 keyword2 keyword3 \keywordnames} % Set the PDF's keywords to your keywords
}

\begin{document}
\renewenvironment{abstract}

%\frontmatter % Use roman page numbering style (i, ii, iii, iv...) for the pre-content pages

\pagestyle{plain} % Default to the plain heading style until the thesis style is called for the body content

%----------------------------------------------------------------------------------------
%	TITLE PAGE
%----------------------------------------------------------------------------------------

\begin{titlepage}
\begin{center}

\vspace*{.06\textheight}
{\scshape\LARGE \univname\par}\vspace{0.5cm} % University name
{\scshape\large \deptname\par}\vspace{1cm} % University name

\includegraphics[width=0.25\textwidth]{images/logo.png} % University/department logo - uncomment to place it
\vspace{1cm}

\textsc{\Large Project proposal}\\[0.5cm] % Thesis type

\HRule\\[0.4cm] % Horizontal line
{\huge \bfseries \ttitle\par}\vspace{0.4cm} % Thesis title
\HRule\\[1.5cm] % Horizontal line

%\begin{minipage}[t]{0.4\textwidth}
\begin{center} \large
\emph{Author:}\\
\href{mailto:f.zappone1@studenti.unimol.it}{\authorname} % Author name - remove the \href bracket to remove the link
\end{center}
%\end{minipage}
%\begin{minipage}[t]{0.4\textwidth}
%\begin{flushright} \large
%\emph{Supervisor:} \\
%\href{Supervisor ref}{\supname} % Supervisor name - remove the \href bracket to remove the link
%\end{flushright}
%\begin{flushright} \large
%\emph{Correlator:} \\
%\href{Correlator ref}{Name Surname} % Correlator name
%\end{flushright}
%\end{minipage}\\[3cm]

%\vfill

\centerline{\large \subjectname}
%\bigskip
{\large December 01, 2020}\\[2cm] % Date
\vfill
\end{center}
\end{titlepage}


%----------------------------------------------------------------------------------------
%	QUOTATION PAGE
%----------------------------------------------------------------------------------------

%\vspace*{0.2\textheight}

%\noindent\enquote{\itshape Thanks to my solid academic training, today I can write hundreds of words on virtually any topic without possessing a shred of information, which is how I got a good job in journalism.}\bigbreak

%\hfill Dave Barry


%----------------------------------------------------------------------------------------
%	ABSTRACT PAGE
%----------------------------------------------------------------------------------------

\begin{abstract}
  \chapter*{Abstract}

\end{abstract}

\vspace*{50px}
\bigskip

%----------------------------------------------------------------------------------------
%	TABLE OF CONTENTS AND LIST OF FIGURES
%----------------------------------------------------------------------------------------

\tableofcontents
\newpage

\listoffigures
\newpage

%----------------------------------------------------------------------------------------
%	PROPOSAL
%----------------------------------------------------------------------------------------

\chapter*{Research topic}

Sempre più spesso si sente parlare di moneta digitale e delle così dette criptovalute, da qualche anno infatti sulla bocca di tutti risaltano parole di innovazione e opportunità in seguito all'avvento della nuova tecnologia blockchain basata sulla logica di database distribuito. Proprio grazie a quest'ultima è nata la prima criptovaluta al mondo, più precisamente, il 3 gennaio 2009 veniva alla luce il blocco genesi di Bitcoin composto allora da 50 \emph{BTC} dal valore complessivo che non raggiungeva neppure un dollaro statunitense dato il prezzo iniziale di \$0.0008 per Bitcoin. In seguito all'ottimo riscontro in molteplici campi il 6 novembre 2010 un Bitcoin si presentava con un valore di \$0,50, in meno di due anni il prezzo era aumentato di 625 volte, e da allora in poco più di sette anni, Bitcoin raggiunse il suo massimo storico di quasi \$20.000 ovvero circa 40.000 volte in più. Divenuto un caso più unico che raro, Bitcoin si è posto da apripista a più di 5000 altre criptovalute fino ad essere definito come l'oro del \RN{21} secolo. Questa definizione non è dovuta solo all'incredibile aumento del prezzo di Bitcoin negli ultimi anni ma anche ad una delle caratteristiche chiave che la maggior parte delle criptovalute condivide, ovvero il mining, il processo di ``estrazione'' delle monete digitali.\\
Il mining di criptovalute consiste nel creare monete virtuali attraverso la risoluzione di alcune funzioni crittografiche necessarie per la validazione delle transazioni e dei blocchi che compongono una blockchain. Questi calcoli vengono eseguiti dai sistemi informatici dedicati a questo specifico processo, questi sistemi sono divisi in due grandi macro sezioni: quelli che sfruttano la Central Processing Unit e quelli che sfruttano invece la Graphics processing unit. I primi prendono il nome di \emph{ASIC} che sta per \emph{Application Specific Integrated Circuit} ovvero circuiti costruiti per la risoluzione di un calcolo ben specifico ma risultano molto inefficienti su altri tipi di algoritmi. I sistemi basati su GPU sono invece molto più prestanti grazie proprio al fatto che le schede video riescono ad effettuare più calcoli al secondo rispetto alle CPU e quindi risultare più redditizie, d'altro canto risultano essere molto più difficili nella gestione delle temperature e nettamente più costose in termini di assemblaggio e di efficienza energetica. Proprio questi ultimi costi sono quelli di notevole impatto quando si parla di mining, il costo di acquisto delle componenti è infatti nettamente aumentato negli ultimi anni, questo sia a causa delle nuove scoperte tecnologiche sia proprio a causa delle grandi farm di mining sparse per il mondo che acquistano sempre più componenti per ingrandire i propri centri di estrazione. Il costo energetico che questi sistemi comportano ha portato invece le grandi farm a svilupparsi maggiormente nei paesi con costi dell'elettricità più bassi e allo stesso tempo ha reso molto più impegnativo l'operazione di mining per chi volesse entrarne a far parte.\\
È così che con l'aumentare dei costi e allo stesso tempo delle opportunità di guadagno offerte dal mondo delle blockchain sempre più in crescita che le criptovalute hanno iniziato a risaltare agli occhi degli hacker malintenzionati.

\chapter*{Related Work}

\chapter*{Aims and Objectives}

\chapter*{Methodology}

%----------------------------------------------------------------------------------------
%	BIBLIOGRAPHY
%----------------------------------------------------------------------------------------

\printbibliography\
\end{document}
